\documentclass[lettersize,journal]{IEEEtran}
\usepackage{amsmath,amsfonts}
\usepackage{algorithmic}
\usepackage{algorithm}
\usepackage{array}
\usepackage[caption=false,font=normalsize,labelfont=sf,textfont=sf]{subfig}
\usepackage{textcomp}
\usepackage{stfloats}
\usepackage{url}
\usepackage{verbatim}
\usepackage{graphicx}
\usepackage{cite}
\hyphenation{op-tical net-works semi-conduc-tor IEEE-Xplore}
% updated with editorial comments 8/9/2021

\begin{document}

\title{SC-407 and MC-226 Project}

\author{Om Gor (202101), Student2 Name (202101),\\ Student N Name (202101), Akshar Panchani (202101522)}

% The paper headers
\markboth{SC-407 Project: Group $18$}%
{Project Title}

% \IEEEpubid{0000--0000/00\$00.00~\copyright~2021 IEEE}
% Remember, if you use this you must call \IEEEpubidadjcol in the second
% column for its text to clear the IEEEpubid mark.

\maketitle

\begin{abstract}
Plastic pollution in oceans has a very significant threat to our marine ecosystems, it's biodiversity, human health and countries golbal economy. As this seems a major concern to conserve and preserve our environment, we highlight the importance of marine life and address the efforts that should be made to reduce the plastic pollution in oceans. 
\end{abstract}

\begin{IEEEkeywords}
Plastic Pollution, Marine Debris, Deep Learning, Autonomous Underwater Vehicles (AUVs), Object Detection, Conservation, Evaluation Metrics
\end{IEEEkeywords}

\section{Introduction}
\IEEEPARstart{P}{lastic} pollution has become a global challenge to many sectors which include food safety, human health, marine life and many more. Despite many efforts applied in this field it still faces many problems such as high cost, inaccurate assessment of plastic waste, labor work to solve such issue, limitation to cover area which is limited to river or lake and does not cover large ocean bodies. Therefore, there is a need to have an innovative solutions to address such issue globally.$\ldots$

\subsection{Problem Statement}
The major problem or the main cause includes the environmental harm to the water bodies in river, lake, sea and ocean, this is because currently more than 14 million tonnes of plastic is deposited in ocean every year causing the depletion of marine ecosystem. The current methods for monitoring marine plastic posses high maintenance, more building cost, also includes greater labor work. Moreover ocean is also polluted due to the rise of carbon dioxide in the environment, leading to acidification of the water. So this bring in a solution to develop a machine learning model with computer vision to detect the plastic precisely and other marine debris throughtout the entire water body.

\subsection{Relevant Works}
Many initiative are taken in this direction to protect our environment. The Government of France has recently implemented the anti-waste law to aim to curb single-use plastic which ends up in freshwater.\cite{France} It is noted that out of 14 million tons of plastic which end up in the ocean, 80\% of the marine debris are found on surface of the water or deep-sea sediments.\cite{IUCN} For this many studies have been conducted such as the remote sensing technique which can be used to conduct aerial surveys on the water bodies to detect floating plastic. \cite{Invest} Researches are made to analize underwater composition of plastic debris on seafloor, this showcase the accumulation of plastic underwater. Many algorithm and use of convolutional neural networks are used to quantify the issue. The SOTA for this is to deploy the autonomous underwater vehicles(AUVs) which are equipped with the sensors and imaging system to monitor the marine plastic debris, they are capable of scanning the large ocean beds. Many collaborative efforts are made by governments and NGOs to provide large datasets and databases to such technology which can detect the plastic in ocean. With the development of this ML model, the software aims to detect plastic in the ocean in order to collect those remains by the AUVs or by manual labor for small water bodies.


\subsection{Our Contributions}
In this study, we use deep learning and computer vision techniques to present a unique strategy to address the problems associated with monitoring marine plastic. Our method seeks to provide an accurate, dependable, and real-time system for measuring and identifying marine plastic waste across the whole water column. Our contribution consists of building on previous research and methods, diversifying datasets, introducing specific modifications, testing multiple object detection models, and assessing model performance with pertinent metrics.


\subsection{Organization}
The structure of this document is as follows. Section II gives a summary of our suggested methodology. The procedures and methodology used are covered in Section III. The experimental findings and assessment metrics are shown in Section IV. In Section V, the work is finally concluded and future directions for research are discussed.

\section{Proposed Approach}

This is the section with the meat of your idea. Here, you describe what you have done. 

It is good to have a general schematic and block diagrammatic representation of your idea. For example, you can include Fig.~\ref{fig_1}, which provides an overview of your proposed approach.

% \begin{figure}[!t]
% \centering
% \includegraphics[width=3.5in]{Fig2.pdf}
% \caption{Simulation results of the proposed approach.}
% \label{fig_1}
% \end{figure}

\subsection{An Analytical Derivation}

\subsubsection{Corollaries}

Organize the content of this section in different subsections, or even subsubsections, to make an impactful and well-organized presentation.

\section{Algorithms}
Algorithms should be numbered and include a short title. They are set off from the text with rules above and below the title and after the last line.

% \begin{algorithm}[H]
% \caption{Weighted Tanimoto ELM.}\label{alg:alg1}
% \begin{algorithmic}
% \STATE 
% \STATE {\textsc{TRAIN}}$(\mathbf{X} \mathbf{T})$
% \STATE \hspace{0.5cm}$ \textbf{select randomly } W \subset \mathbf{X}  $
% \STATE \hspace{0.5cm}$ N_\mathbf{t} \gets | \{ i : \mathbf{t}_i = \mathbf{t} \} | $ \textbf{ for } $ \mathbf{t}= -1,+1 $
% \STATE \hspace{0.5cm}$ B_i \gets \sqrt{ \textsc{max}(N_{-1},N_{+1}) / N_{\mathbf{t}_i} } $ \textbf{ for } $ i = 1,...,N $
% \STATE \hspace{0.5cm}$ \hat{\mathbf{H}} \gets  B \cdot (\mathbf{X}^T\textbf{W})/( \mathbb{1}\mathbf{X} + \mathbb{1}\textbf{W} - \mathbf{X}^T\textbf{W} ) $
% \STATE \hspace{0.5cm}$ \beta \gets \left ( I/C + \hat{\mathbf{H}}^T\hat{\mathbf{H}} \right )^{-1}(\hat{\mathbf{H}}^T B\cdot \mathbf{T})  $
% \STATE \hspace{0.5cm}\textbf{return}  $\textbf{W},  \beta $
% \STATE 
% \STATE {\textsc{PREDICT}}$(\mathbf{X} )$
% \STATE \hspace{0.5cm}$ \mathbf{H} \gets  (\mathbf{X}^T\textbf{W} )/( \mathbb{1}\mathbf{X}  + \mathbb{1}\textbf{W}- \mathbf{X}^T\textbf{W}  ) $
% \STATE \hspace{0.5cm}\textbf{return}  $\textsc{sign}( \mathbf{H} \beta )$
% \end{algorithmic}
% \label{alg1}
% \end{algorithm}

\section{Discussion and Remarks}

Provide conceptual insights here. You can discuss alternative approaches that you may have investigated. You can make remarks on any relevant understandings and studies you did. Through this section, you show the reader that you have gone to a great depth in solving the problem.

\section{Numerical Results}

Fig.~\ref{fig_2} provides an experimental evaluation of our proposed methodology.

% \begin{figure}[!t]
% \centering
% \includegraphics[width=3.5in]{Fig7.pdf}
% \caption{Simulation results of the proposed approach.}
% \label{fig_2}
% \end{figure}

\section{Conclusion}
Finally, we have suggested a novel strategy for reducing ocean plastic waste that makes use of deep learning and computer vision methods. Our strategy seeks to overcome the drawbacks of conventional marine plastic monitoring techniques and offer an affordable, on-the-spot solution for accurate marine plastic waste identification and quantification.


{\appendices
\section{Proof of the First Zonklar Equation}
Appendix one text goes here.

\bibliographystyle{ieeetr}
\bibliography{sc407bib}

\end{document}


